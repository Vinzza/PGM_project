\documentclass[a4paper,11pt,titlepage]{article}
\usepackage[utf8]{inputenc}
\usepackage[T1]{fontenc}
\usepackage{lmodern}
\usepackage[french]{babel}
\usepackage{amsmath, amsfonts, amssymb, amsthm, dsfont}
\usepackage{bbold}
\usepackage{stmaryrd}
\usepackage{mathrsfs}
\usepackage{fancyhdr}
\usepackage{color}
\usepackage[colorlinks=true,linkcolor=blue]{hyperref}
\newtheorem{mydef}{Definition}

\begin{document}
\section{Syntaxe git}
\begin{itemize}
	\item 0. \textbf{git log} Pour regarder les derniers commits, notemment s'il y en a eu ou non, pour s'épargner un pull (pas recommandé).
	\item 1. \textbf{git pull} Toujours pull pour vérifier qu'on a la dernière version. (au pire travailler avec des fichiers temporaires sur un autre dossier pour éviter les conflits)
	\item 2. \textbf{git add file1 file2 file3 \dots} On ajoute les fichiers qu'on veut updater sur le serveur. Il faut se placer dans le dossier où ils se situent pour ça (voir commandes linux en \ref{linux})
	\item 3. \textbf{git commit} Là il t'ouvre un editeur a priori pour que tu rédiges un résumé sommaire de ce que tu as fait à destination des autres.
	\item 4. \textbf{git push} On envoie le tout au serveur. Le tour est joué.
\end{itemize}

\section{Commandes linux utiles}
\label{linux}
\begin{itemize}
	\item \textbf{cd dir} ``change directory'' : Va dans le dossier dir. Tu peux donc préciser le dossier depuis sa localisation dans windows (C/home/nathan/Documents/\dots) ou par rapport à là où tu es (juste le nom du dossier s'il est dans le dossier où tu te trouves. Ex : cd Code).
	\item \textbf{ls} ``list'' : Qu'y a t'il dans le dossier où je me trouve justement (dossiers et fichiers)?
	\item \textbf{pwd} ``print working directory'' : Où suis-je d'ailleurs ? En pratique c'est toujours affiché dans la console au dessus du prompt (le chevron > ).
\end{itemize}
Le reste ne te concerne pas puisque tu peux, sous windows, renommer/deplacer/supprimer un fichier sans le faire dans la console, donc restons en là pour l'instant.
\end{document}

