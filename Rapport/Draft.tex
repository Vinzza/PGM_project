\documentclass[10pt, a4paper, twocolumn]{article}


%% %%%%%%%%%%%%%%%%%%%%%%%%%%%%%%%%%%%%%%%%%%%%%%%%%%%%%%%%%%%%%%%%%%%%%%%%%%%%%
%% %%%%%%% %% %% %  %                                      %  % %% %% %%%%%%%%%%
%%%%% %% %  %                    MES COMMANDES                     %  % %% %%%%%
%%%%%%%%%% %% %% %  %                                      %  % %% %% %%%%%%% %%
%%%%%%%%%%%%%%%%%%%%%%%%%%%%%%%%%%%%%%%%%%%%%%%%%%%%%%%%%%%%%%%%%%%%%%%%%%%%% %%

\newcommand{\zzpackages}[1][french]{
  \usepackage[utf8]{inputenc}
  \usepackage[T1]{fontenc}
  \usepackage[#1]{babel}

  \usepackage{amsthm}
  \usepackage{amsmath}
  \usepackage{amsfonts}
  \usepackage{amssymb}

  \usepackage{xcolor}
  \usepackage{xstring}	%\ifstreqcase
}

%% %%%%%%%%%%%%%%%%%%%%%%%%%%%%%%%%%%%%%%%%%%%%%%%%%%%%%%%%%%%%%%%%%%%%%%%%%%%%%
%                              TESTS / FOR / ...                               %
%%%%%%%%%%%%%%%%%%%%%%%%%%%%%%%%%%%%%%%%%%%%%%%%%%%%%%%%%%%%%%%%%%%%%%%%%%%%% %%

% 
\def\defactive#1#2{
  \catcode`#1=13
  \begingroup
  \lcode`~=`#1
  \lowercase{\endgroup\def~}{#2}
}

\def\zifempty#1#2#3{\def\foo{#1}\ifx\foo\empty\relax#2\else#3\fi}

%% %%%%%%%%%%%%%%%%%%%%%%%%%%%%%%%%%%%%%%%%%%%%%%%%%%%%%%%%%%%%%%%%%%%%%%%%%%%%%
%                           MARGES / HYPERREF / ...                            %
%%%%%%%%%%%%%%%%%%%%%%%%%%%%%%%%%%%%%%%%%%%%%%%%%%%%%%%%%%%%%%%%%%%%%%%%%%%%% %%

\makeatletter
\gdef\@subtitle{}
\def\subtitle#1{\gdef\@subtitle{#1}}

\def\zztitre{
\begingroup\centering
{\bfseries \huge \@title}\par\vspace{.3cm}
\ifx\@subtitle\empty\else{\bfseries \Large \@subtitle}\par\vspace{.5cm}\fi
\Large \@author\par\vspace{.1cm}
\@date\zal\vspace{.3cm}\zal
\zligne\endgroup
}

\makeatother


\newcommand{\zzhyperref}{
\usepackage{hyperref}
\hypersetup{ colorlinks=true, linkcolor=blue!30!black,
citecolor=green!30!black, filecolor=magenta!30!black,
urlcolor=cyan!30!black }
}

\newcommand{\zzmarges}{
  \setlength{\textheight}{620pt}
  \addtolength{\textwidth}{2cm}
  \addtolength{\hoffset}{-1cm}
  \addtolength{\voffset}{-1cm}
  \addtolength{\marginparwidth}{0cm}
  \addtolength{\textheight}{1cm}
} 

\makeatletter
\newcommand{\zzheader}[6]{
\def\@oddhead{\vbox to 0pt{\vss\hspace{0pt} #1\hfill #2\hfill #3\kern4pt\par\kern5pt\hrule height.5pt}}
\def\@oddfoot{\vbox to 0pt{\hrule height.5pt\kern5pt\hbox to \linewidth{\kern4pt {#4}\hss {#5}\hss {#6}\kern4pt}\vss}}}
\makeatother

%%%%%%%%%%%%%%%%%%%%%%%%%%%%%%%%%%%%%%%%%%%%%%%%%%%%%%%%%%%%%%%%%%%%%%%%%%%%%%%%

\newcommand{\zligne}[1][]{
\par\zifempty{#1}%
{\hbox to \linewidth{\leaders\hrule height3pt depth-2.5pt\hfill}}%
{\hbox to \linewidth{\leaders\hrule height3pt depth-2.5pt\hfill\kern.8em #1\kern.8em\leaders\hrule height3pt depth-2.5pt\hfill}}\par
}

%%%%%%%%%%%%%%%%%%%%%%%%%%%%%%%%%%%%%%%%%%%%%%%%%%%%%%%%%%%%%%%%%%%%%%%%%%%%%%%%

\newcommand{\zal}{\par}
\newcommand{\znl}{\zal ~\zal}
\newcommand{\zguill}[2][]{«\,#2\,»}

%% %%%%%%%%%%%%%%%%%%%%%%%%%%%%%%%%%%%%%%%%%%%%%%%%%%%%%%%%%%%%%%%%%%%%%%%%%%%%%
%                             COMMANDES PRATIQUES                              %
%%%%%%%%%%%%%%%%%%%%%%%%%%%%%%%%%%%%%%%%%%%%%%%%%%%%%%%%%%%%%%%%%%%%%%%%%%%%% %%

%% ------- -- -- -  -                                      -  - -- -- --------%%
%---- -- -  -                          ZP                          -  - -- ----%
%%-------- -- -- -  -                                      -  - -- -- ------- %%
\makeatletter
\def\z@first#1#2{#1}
\def\z@second#1#2{#2}
\def\z@zp@selectchar#1#2{
  \IfStrEqCase{#2}{%
    {p}{#1{(}{)}}%
    {c}{#1{[}{]}}%
    {a}{#1{\{}{\}}}%
    {C}{#1{]}{[}}%
    {b}{#1{|}{|}}%
    {n}{#1{\|}{\|}}%
    {i}{#1{[}{]}\!#1{[}{]}}%
    {t}{#1{<}{>}}%
    {v}{#1{.}{.}}%
    {A}{#1{\}}{\{}}%
    {P}{#1{)}{(}}%
    {I}{#1{]}{[}\!#1{]}{[}}%
    {T}{#1{>}{<}}%
  }[#1{(}{)}]%
}

\def\z@zp#1#2\fin#3{
  \z@zp@selectchar{\left\z@first}{#1}#3
  \zifempty{#2}%
        {\z@zp@selectchar{\right\z@second}{#1}}%
        {\z@zp@selectchar{\right\z@second}{#2}}%
}
\newcommand{\zp}[2][]{\zifempty{#1}{\left(#2\right)}{\z@zp#1\fin{#2}}}

\newcommand{\zpbig}[1]{\ifcase#1\relax\vrule width0pt height0pt\or\vrule width0pt height9pt\or\vrule width0pt height10pt\or\vrule width0pt height13pt\else\vrule width0pt height16pt\fi}

%% ------- -- -- -  -                                      -  - -- -- --------%%
%---- -- -  -                   Itemize et autre                   -  - -- ----%
%%-------- -- -- -  -                                      -  - -- -- ------- %%

\newcommand{\zitemize}[1]{
\vspace{-\topsep}\begin{itemize}\setlength\itemsep{0pt plus 1pt}\setlength\parskip{0cm}#1\end{itemize}\vspace{-\topsep}}


%% ------- -- -- -  -                                      -  - -- -- --------%%
%---- -- -  -                        AUTRES                        -  - -- ----%
%%-------- -- -- -  -                                      -  - -- -- ------- %%

\newcommand{\zR}{\mathbb{R}}

\newcommand{\zsum}[2][0pt]{\sum_{\hbox to #1{\hss$\scriptstyle#2$\hss}}}
\newcommand{\zprod}[2][0pt]{\prod_{\hbox to #1{\hss$\scriptstyle#2$\hss}}}

\newcommand{\zseq}[1][=]{\hspace{2pt}\raise .5pt\hbox{\scalebox{.8}{#1}}\hspace{2pt}}

\newcommand{\zop}[2]{\mathrm{#1}\zp{#2}}

\newcommand{\zi}{\mathrm{i}}

\newcommand{\zexp}[1]{\mathrm{e}^{#1}}

\newcommand{\zmatrix}[2]{\left(\begin{array}{#1}#2\end{array}\right)}

\newcommand{\zindic}[1]{%
\hbox to 5.3pt{1\hss l}\hskip -2.5pt\left\{#1\right\}%
}

\newcommand{\zesp}[2][]{%              esperance
\mathbb{E}_{#1}\hskip -3pt\left[\zpbig1\,#2\,\right]%
}

\newcommand{\zprob}[2][]{%             proba
\mathbb{P}_{#1}\hskip -3pt\left(\zpbig1\,#2\,\right)%
}

% Symbole d'indépendance de variable aléatoire
\newcommand{\zindep}{\protect\mathpalette{\protect\z@ind}{\perp}}
\def\z@ind#1#2{\mathrel{\rlap{$#1#2$}\mkern6mu{#1#2}}}


\newcommand{\zdx}[1]{\mathrm{d}#1}

\newcommand{\zderiv}[2]{\frac{\partial #1}{\partial #2}}


\newcommand{\ztr}[2][]{\zifempty{#1}{#2}{\left(#2\right)}^{\hspace{-1pt}\mathsf{T}}\hspace{-1pt}}
\def\zpreind#1#2{ \raise-.35ex\hbox{\scriptsize$#1$}#2}
\def\zpreexp#1#2{ \raise.85ex\hbox{\scriptsize$#1$}#2}

%% %%%%%%%%%%%%%%%%%%%%%%%%%%%%%%%%%%%%%%%%%%%%%%%%%%%%%%%%%%%%%%%%%%%%%%%%%%%%%
%                                 ALGORITHMES                                  %
%%%%%%%%%%%%%%%%%%%%%%%%%%%%%%%%%%%%%%%%%%%%%%%%%%%%%%%%%%%%%%%%%%%%%%%%%%%%% %%

\newcount\z@algo@count
\newdimen\z@algo@indent
\begingroup
  \catcode`\^^M=13             %
  \catcode`\^^I=13             %
  \gdef\z@algo{                %
  \z@algo@count=1
    \begingroup                %
    \catcode`\^^M=13           %
    \def^^M{\leavevmode\par \advance\z@algo@count by 1\z@algo@indent=0pt}%
    \catcode`\^^I=13           %
    \def^^I{\advance\z@algo@indent by 1em}         %
    \everypar{                 %
      \hbox to 0cm{\hss\textcolor{black!30}{\the\z@algo@count~:}}~\kern\z@algo@indent}  %
                               %
    \tt                        %
  }

\endgroup

\newenvironment{zalgo}{\z@algo}{\endgroup}


\makeatother



%% %%%%%%%%%%%%%%%%%%%%%%%%%%%%%%%%%%%%%%%%%%%%%%%%%%%%%%%%%%%%%%%%%%%%%%%%%%%%%
%                                    AUTRE                                     %
%%%%%%%%%%%%%%%%%%%%%%%%%%%%%%%%%%%%%%%%%%%%%%%%%%%%%%%%%%%%%%%%%%%%%%%%%%%%% %%

% epaisseur trait / marge / texte

\def\zfbox#1#2#3{
  \hbox{\vrule width #1
    \vtop{
      \vbox{
        \hrule height #1
        \kern #2
        \hbox{\kern #2 #3\kern #2}
      }%
      \kern #2%
      \hrule height #1
    }%
    \vrule width #1%
  }%
}

%% \begin{mygraph}{xmin=0, xmax=1, %
%%                ymin=0, ymax=1, %
%%                sizex=2.5, sizey=2.5}%
%%                {nomx=Axe X, nomy=Axe Y}%
%%                {0,.5,1}{0,0.25,...,1.05}

%% \graduationX[dashed, blue]{ .78 / $\frac{\pi}{4}$ }{ PARAMETRE TEXT }

%% \begin{mylegend}{x=0.3, y=.9, n=2, t=2.1, scale=.5}
%%   \newlegend{blue}{Courbe 1}
%%   \newlegend{red}{Courbe 2}
%% \end{mylegend}

%% \fillbetweencurve[opacity=.2, blue]{ COURBE 1 }{ COURBE 2 }

%% \end{mygraph}


\usepackage{tikz}



\pgfkeys{
%
 /mygraph/.is family, /mygraph,
 xmin/.estore in = \xn,
 xmax/.estore in = \xm,
 ymin/.estore in = \yn,
 ymax/.estore in = \ym,
 sizex/.estore in = \xx,
 sizey/.estore in = \yy,
 %
/mygraphb/.is family, /mygraphb,
 nomx/.estore in = \axex,
 nomy/.estore in = \axey,
%
/mygraphc/.is family, /mygraphc,
 gradsize/.estore in = \gradsize,
 gradsize/.default = 0.1,
 nomydist/.estore in = \axeyd,
 nomydist/.default = 0.8cm,
 gradsize, nomydist,                 % NE PAS OUBLIER
%
/myleg/.is family, /myleg,
 x/.estore in = \legendx,
 y/.estore in = \legendy,
 n/.estore in = \legendn,
 t/.estore in = \legendt,
 scale/.estore in = \legends,
 scale/.default = 1,
 scale,
%
/mylego/.is family, /mylego,
size/.estore in = \legendwidth,
size/.default = 0.4,
size                                  % NE PAS OUBLIER
}


%%%%%%%%%%%%%%%%%%%%%%%%%%%%%%%%%%%%%%%%%%%%%%%%%%%%%%%%%%%%%%%%%%%%%%%%%%%%%%%%
%                                                                              %
%%%%%%%%%%%%%%%%%%%%%%%%%%%%%%%%%%%%%%%%%%%%%%%%%%%%%%%%%%%%%%%%%%%%%%%%%%%%%%%%

\newenvironment{mygraph}[5][]{%
\pgfkeys{/mygraph, #2}
\pgfkeys{/mygraphb, #3}
\pgfkeys{/mygraphc, #1}
  \pgfmathsetmacro\dum{\yy/(\ym-\yn)}
  \pgfmathsetmacro\dumm{\xx/(\xm-\xn)}
\begin{tikzpicture}[yscale=\dum, xscale=\dumm,font=\sffamily]
  \pgfmathsetmacro\gradx{\gradsize / \dum}
  \pgfmathsetmacro\grady{\gradsize / \dumm}

  \foreach \x in {#4}{
    \draw[very thin, color=black, dotted] (\x,\yn) -- (\x,\ym);
    \draw (\x,\yn+\gradx) -- (\x,\yn)
          node[font=\tiny, anchor=north] {\pgfmathprintnumber{\x}};
  };
  \foreach \y in {#5}{
    \draw[very thin, color=black, dotted] (\xn,\y) -- (\xm,\y); 
    \draw (\xn+\grady,\y) -- (\xn,\y)
          node[font=\tiny, anchor=east] {\pgfmathprintnumber{\y}};
  };
  \draw (\xn,\yn) -- node[font=\scriptsize, below=0.3cm] {\axex} (\xm,\yn);
  \draw (\xn,\yn) -- node[font=\scriptsize, rotate=90, above=\axeyd, anchor=mid] {\axey} (\xn,\ym);
  \draw (\xn,\ym) -- (\xm,\ym);
  \draw (\xm,\yn) -- (\xm,\ym);

  \begin{scope}
    \clip (\xn,\yn) rectangle (\xm,\ym);
    %% \draw[dashed] (\xn, 0) -- (\xm, 0);
    %% \draw[dashed] (0, \yn) -- (0, \ym);
}{
  \end{scope}
\end{tikzpicture}
}

%%%%%%%%%%%%%%%%%%%%%%%%%%%%%%%%%%%%%%%%%%%%%%%%%%%%%%%%%%%%%%%%%%%%%%%%%%%%%%%%

\newenvironment{mylegend}[2][]{
\pgfkeys{/myleg, #2}
\pgfkeys{/mylego, #1}
\begin{scope}[shift={(\legendx,\legendy)}, scale=\legends]
\pgfmathsetmacro\legendwidth{\legendwidth * (\xm-\xn) / \xx }
\pgfmathsetmacro\dum{ (0.125 * (\ym-\yn) / \yy) }
\pgfmathsetmacro\dumm{ - (0.125 * (\xm-\xn) / \xx) }
\pgfmathsetmacro\legendy{ 0 }
\coordinate (dum) at (\dumm,\dum);
\pgfmathsetmacro\dum{ - (\legendn-0.4)*(0.25 * (\ym-\yn) / \yy) }
\coordinate (dumm) at (\dumm,\dum);
\pgfmathsetmacro\dumm{\dumm + (\legendt * (\xm-\xn) / \xx) }
\draw[fill=white, opacity=.8] (dum) -- (dumm) -| (\dumm,\dum) |- (dum);
}{
\end{scope}
%% \pgfmathsetmacro\legendyi{\legendyi + (0.125 * (\ym-\yn) / \yy)  }
%% \pgfmathsetmacro\legendy{\legendy + (0.1 * (\ym-\yn) / \yy)  }
%% \pgfmathsetmacro\legendx{\legendx - (0.125 * (\xm-\xn) / \xx)  }
%% \draw[] (\legendx,\legendyi) -- (\legendx,\legendy) %
%%                              -| (\legendx + 1,\legendy)
%%                              |- (\legendx,\legendyi);
}

\newcommand{\newlegend}[2]{
\draw[font=\scriptsize, #1] (0,\legendy) -- (\legendwidth,\legendy)	node[right,scale=\legends]{#2};
\pgfmathsetmacro\legendy{\legendy - (0.25 * (\ym-\yn) / \yy)  }
}

%%%%%%%%%%%%%%%%%%%%%%%%%%%%%%%%%%%%%%%%%%%%%%%%%%%%%%%%%%%%%%%%%%%%%%%%%%%%%%%%

\newenvironment{outofbox}{%
  \end{scope}%
}{%
  \begin{scope}%
    \clip (\xn,\yn) rectangle (\xm,\ym);%
}

%%%%%%%%%%%%%%%%%%%%%%%%%%%%%%%%%%%%%%%%%%%%%%%%%%%%%%%%%%%%%%%%%%%%%%%%%%%%%%%%

\newcommand{\fillbetweencurve}[3][]{
\begin{scope}
\clip (\xn,\yn) -- #2 -- (\xm,\yn) -- cycle;
\fill[#1] (\xn,\ym) -- #3 -- (\xm,\ym) -- cycle;
\end{scope}
}

%%%%%%%%%%%%%%%%%%%%%%%%%%%%%%%%%%%%%%%%%%%%%%%%%%%%%%%%%%%%%%%%%%%%%%%%%%%%%%%%

\newcommand{\graduationX}[3][very thin, color=black, dotted]{
\end{scope}
  \foreach \x/\t in {#2}{
    \draw[#1] (\x,\yn) -- (\x,\ym);
    \draw (\x,\yn+\gradx) -- (\x,\yn)
          node[font=\tiny, anchor=north, #3] {\t};
  };
\begin{scope}%
\clip (\xn,\yn) rectangle (\xm,\ym);%
}

%%%%%%%%%%%%%%%%%%%%%%%%%%%%%%%%%%%%%%%%%%%%%%%%%%%%%%%%%%%%%%%%%%%%%%%%%%%%%%%%

\newcommand{\graduationY}[3][very thin, color=black, dotted]{
\end{scope}
  \foreach \y/\t in {#2}{
    \draw[#1] (\xn,\y) -- (\xm,\y);
    \draw (\xn+\grady,\y) -- (\xn,\y)
          node[anchor=east, font=\tiny, shift={(-0*\grady,0)}, #3] {\t};
  };
\begin{scope}%
\clip (\xn,\yn) rectangle (\xm,\ym);%
}


\zzpackages[english]


%% \setlength{\textwidth}{460pt}
%% \setlength{\hoffset}{-54pt}
\setlength{\textwidth}{495pt}
\setlength{\hoffset}{-20pt}
\addtolength{\textheight}{1.6in}
\setlength{\voffset}{-60pt}

%\zzheader{a}{b}{c}{d}{e}{f}

%% %%%%%%%%%%%%%%%%%%%%%%%%%%%%%%%%%%%%%%%%%%%%%%%%%%%%%%%%%%%%%%%%%%%%%%%%%%%%%
%                                MACRO LOCALES                                 %
%%%%%%%%%%%%%%%%%%%%%%%%%%%%%%%%%%%%%%%%%%%%%%%%%%%%%%%%%%%%%%%%%%%%%%%%%%%%% %%

\newcommand{\mysection}[1]{\vspace{-0pt}\section{#1}\vspace{0pt}}
\newcommand{\mysubsection}[1]{\vspace{-7pt}\subsection{\normalsize #1}\vspace{-2pt}}


\newcommand{\Kl}[3][]{\mathrm K_{#1}\!\zp{#2\:\|\:#3}}
\newcommand{\zZ}[2]{\mathrm #1\!\zp{#2}}
\newcommand{\zD}{\mathcal}
\newcommand{\Ng}[2]{\mathcal{N}\zp{#1,\:#2}}

\tikzset{
  zplot/.style={opacity=.8}
}

%% %%%%%%%%%%%%%%%%%%%%%%%%%%%%%%%%%%%%%%%%%%%%%%%%%%%%%%%%%%%%%%%%%%%%%%%%%%%%%
%                                    TITRE                                     %
%%%%%%%%%%%%%%%%%%%%%%%%%%%%%%%%%%%%%%%%%%%%%%%%%%%%%%%%%%%%%%%%%%%%%%%%%%%%% %%

\begin{document}

\allowdisplaybreaks[1]
 \twocolumn[{
~\vspace{-30pt}

\begin{center}
  \textbf{ {\huge Independent Component Analysis} \\[5pt]
{\Large MVA Project -- One page draft}}\\[15pt]
  {\large \textsc{Nathan de Lara\hss Florian Tilquin\hss Vincent Vidal}}
\end{center}
\vspace{-10pt}

\zligne

\vspace{.2cm}

%% %%%%%%%%%%%%%%%%%%%%%%%%%%%%%%%%%%%%%%%%%%%%%%%%%%%%%%%%%%%%%%%%%%%%%%%%%%%%%
%                                   ABSTRACT                                   %
%%%%%%%%%%%%%%%%%%%%%%%%%%%%%%%%%%%%%%%%%%%%%%%%%%%%%%%%%%%%%%%%%%%%%%%%%%%%% %%

\hbox to \textwidth{\hss \textsc{Abstract}\hss}
We decided to work on the Independent Component Analysis problem. We intent to implement, apply and compare several algorithms while being certain to understand the link between the likelihood maximisation and the mutual information.
\vspace{.7cm}
%
%% Ne pas toucher ça (en dessous)
}]
%% %%%%%%%%%%%%%%%%%%%%%%%%%%%%%%%%%%%%%%%%%%%%%%%%%%%%%%%%%%%%%%%%%%%%%%%%%%%%%
%                              DEBUT DU DOCUMENT                               %
%%%%%%%%%%%%%%%%%%%%%%%%%%%%%%%%%%%%%%%%%%%%%%%%%%%%%%%%%%%%%%%%%%%%%%%%%%%%% %%

\mysection{Definition of the problem}

The general problem can be formalised this way.
Suppose we have some random variables $x\in\zR^p$ which correspond to
a mix of some primitive sources $s\in\zR^n$. The aim is to extract
from $x$ every source $s_i$. To do so, we will suppose here that:\\
-- the sources are independents\\
-- the mix is linear and instantaneous\\
-- at most one source has a gaussian distribution.\\
We will write:
\begin{equation}
  x = A s \ \ \mbox{and} \ \ y = W x,
\end{equation}
where $A$ is the mixing matrix, $W$ the separation matrix and $y$ the
estimation of the sources. The goal is then to find the matrix $W$ that
maximise the independence of $y$.
%% But because only $x$ is known, the problem is to find a matrix
%% $A\in\zR^{p\times n}$ of rank $n$ and a positive diagonal matrix $D$ such
%% that the covariance matrix of $x$ can be written $A D^2 \ztr A$ and
%% \hbox{$x = A s$} with $s$ of covariance $D^2$ and such that the component
%% of $s$ are as most independent as possible. 
As a measure of independence, we will consider, for theoretical purpose,
the mutual information:\begin{equation}
  \zZ IY = \int_{\zR^p} P(Y) \log\frac{P(Y)}{\Pi_i P_i(Y_i)} \zdx Y.
\end{equation}
However, as it is too hard to compute, we will consider other contrast functions, invariant by permutation, scaling on coordinates and maximal for independant ones.
%For practical purpose, we will consider other contrast function than the
%mutual information (too hard to compute), which are invariant by permutation
%and scaling on coordinates and ``minimal'' for independent coordinates.


%% ------- -- -- -  -                                      -  - -- -- --------%%
%---- -- -  -                     Algorithmes                      -  - -- ----%
%%-------- -- -- -  -                                      -  - -- -- ------- %%

\mysection{Algorithms for ICA}
\mysubsection{Hérault and Jutten (HJ) algorithm}
This method is based on the neural network principle.
We write $W = \zp{I_n+\widetilde W}^{-1}$ and for a pair of given functions $\zp{f,\:g}$, we adapt the $\widetilde W$ as follows:\begin{equation}
\widetilde W_{ij} = f(y_i) g(y_j).
\end{equation}

%% --------- ---- -  -            -----------             -  - ---- --------- %%

\mysubsection{EASI algorithm}
For a given contrast function $\psi$ and a cost function $J_\psi$, we iterate as follows:
\begin{equation}
  W_{t+1} = \zp{I_n-\lambda_t\nabla J_\psi(y_t)} W_t.
\end{equation}
The EASI algorithm correspond to the functions \hbox{$\psi(y) = \sum \zp[b]{y_i}^4$} and $J_\psi(y) = \zesp{\psi(y)}$.

%% --------- ---- -  -            -----------             -  - ---- --------- %%

\mysubsection{Jade algorithm}
Several methods are based on the cumulants. The aim here is to annul all the cross cumulants of order $4$.
Thus, we diagonalize the cumulant tensor which is equivalent to minimise the following contrast function:
\begin{equation}
  c\zp{x} = \sum_{i,k,l}\zp[b]{\zop{Cum}{x_i,x_i^*,x_k,x_l}}^2.
\end{equation}

%% --------- ---- -  -            -----------             -  - ---- --------- %%

\mysubsection{Infomax algorithm}

CHANGER PAR FASTICA ?
%%This method is based on the information theory equality
%%\hbox{$\zderiv{\zZ I{y,\:x}}w = \zderiv{\zZ Hy}w$},
%%which gives us an update rule for minimising the mutual information:
%%\begin{equation}
%%  \Delta W=\zp[c]{\zpbig2\ztr W}^{-1}+\zderiv{\ln \Pi_i\zp[b]{y'_i}}W.
%%\end{equation}


%% ------- -- -- -  -                                      -  - -- -- --------%%
%---- -- -  -                      Resultats                       -  - -- ----%
%%-------- -- -- -  -                                      -  - -- -- ------- %%

\mysection{Results}
First, we decided to apply the main algorithm to simple simulated data.
For three sinusoidal signals of arbitrary frequencies and phases, we obtain the results in figure \ref{fig:res}.

\begin{figure}
\hbox{\hspace{-.8cm}\begin{mygraph}{xmin=0, xmax=100, %
                ymin=-1, ymax=1, %
                sizex=3.4, sizey=1.5}%
                {nomx=Sources, nomy=}%
                {0,50,100}{-1,-.5,...,1.05}
  \draw[zplot, red] plot file {data/init_sources_2.txt};
  \draw[zplot, green!70!black] plot file {data/init_sources_3.txt};
  \draw[zplot, blue] plot file {data/init_sources_1.txt};
\end{mygraph}\hspace{-1cm}
\begin{mygraph}{xmin=0, xmax=100, %
                ymin=-1, ymax=1, %
                sizex=3.4, sizey=1.5}%
                {nomx=FastICA, nomy=}%
                {0,50,100}{-1,-.5,...,1.05}
  \draw[zplot, red] plot file {data/init_fast_1.txt};
  \draw[zplot, green!70!black] plot file {data/init_fast_2.txt};
  \draw[zplot, blue] plot file {data/init_fast_3.txt};
\end{mygraph}\hss}
\hbox{\hspace{-.8cm}\begin{mygraph}{xmin=0, xmax=100, %
                ymin=-1, ymax=1, %
                sizex=3.4, sizey=1.5}%
                {nomx=HJ, nomy=}%
                {0,50,100}{-1,-.5,...,1.05}
  \draw[zplot, red] plot file {data/init_hj_2.txt};
  \draw[zplot, green!70!black] plot file {data/init_hj_1.txt};
  \draw[zplot, blue] plot file {data/init_hj_3.txt};
\end{mygraph}\hspace{-1cm}
\begin{mygraph}{xmin=0, xmax=100, %
                ymin=-1, ymax=1, %
                sizex=3.4, sizey=1.5}%
                {nomx=JADE, nomy=}%
                {0,50,100}{-1,-.5,...,1.05}
  \draw[zplot, red] plot file {data/init_jade_3.txt};
  \draw[zplot, green!70!black] plot file {data/init_jade_1.txt};
  \draw[zplot, blue] plot file {data/init_jade_2.txt};
\end{mygraph}\hss}
\caption{Results of the main ICA algorithms on simulated data. Each colour corresponds to a signal. The ``Sources'' graph shows the unmixed signals and the other ones the results of the algorithms. \label{fig:res}}
\end{figure}

\mysection{What's next}

We intent now to implement our own version of some of the algorithms and apply it to different types of data such as electroencephalogram, financial data and images. We will also look into more recent methods such as kernel ICA.

%% \bibliographystyle{plain}
%% \bibliography{Biblio}{}
%% \nocite{*}
%% \label{lastpage}

\end{document}


