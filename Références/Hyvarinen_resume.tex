\documentclass[a4paper,twoside,10pt]{article}

\setlength{\oddsidemargin}{9pt} % Marge gauche sur pages impaires
\setlength{\evensidemargin}{9pt} % Marge gauche sur pages paires
\setlength{\textwidth}{460pt} % Largeur de la zone de texte (17cm)
\setlength{\topmargin}{0pt} % Pas de marge en haut
\setlength{\headheight}{13pt} % Haut de page
\setlength{\headsep}{10pt} % Entre le haut de page et le texte
\setlength{\textheight}{660pt} % Hauteur de la zone de texte (25cm)

\usepackage[francais, english]{babel}
\usepackage[T1]{fontenc}
\usepackage[ansinew]{inputenc}
\usepackage{bbold}
\usepackage{lmodern}
\newcommand{\argmin}{\arg\!\min}
\usepackage{algorithm}

\usepackage{graphicx}
\usepackage{amsmath}
\usepackage{amsthm}
\usepackage{amsfonts}
\usepackage{algorithm}
\usepackage[noend]{algpseudocode}

\begin{document}

\pagestyle{empty}

\title{Syntheses Hyvarinen}

\author{Nathan de Lara}
%\date{}
\maketitle

\section{Results}
\paragraph{ICA}
Assume that we observe $n$ linear mixtures $x_1, ...,x_n$ of n independent components:
\begin{align*}
x_j = a_{j1}s_1+a_{j2}s_2+...+a_{jn}s_n, \forall j
\end{align*}
where all $x_j$, $s_j$ are random variables with zero mean and the $s_j$ are statistically independent. Equivalently:
\begin{align*}
x = As
\end{align*}
Given the entry $x$, the goal is to estimate both $A$ and $s$.

\paragraph{Constraint} The vector $s$ must be \textit{nongaussian}. One can prove that the distribution of any orthogonal transformation of the Gaussian $(x_1,x_2)$
has exactly the same distribution as $(x_1,x_2)$, and that $x_1$ and $x_2$ are independent. Thus, in the case of Gaussian variables, we can only estimate the ICA model up to an orthogonal transformation. In other words, the matrix A is not identifiable for Gaussian independent components.

\paragraph{Notations} $y = w^Tx$ and $z=A^Tw$ such that: $y = z^Ts$

\paragraph{Idea} Use the Central Limit Theorem to determine $w$ so that it would
equal one of the rows of the inverse of $A$. Choose $w$ such that it corresponds to a $z$ with only one nonzero coordinate. Consequently, $z^Ts$ is equal to one of the wanted components up to a multiplicative sign.

\paragraph{Kurtosis} This quantity defined for a zero-mean random variable with unit variance by:
\begin{align*}
kurt(y) &= \mathbb{E}(y^4)-3(\mathbb{E}(y^2))^2\\
\end{align*}
is commonly used to measure nongaussianity as is it zero for a Gaussian variable and often nonzero for a nongaussian variable. Besides, for independent variables:
\begin{align*}
kurt(x_1+x_2) &= kurt(x_1) + kurt(x_2)\\
kurt(\alpha x_1) &= \alpha^4 kurt(x_1)
\end{align*}

\paragraph{Problem formulation}
\begin{align*}
max&imize \enspace |kurt(y)|\\
&st \enspace y = z^Ts\\
an&d \enspace Var(y)=1
\end{align*}
The drawback is that the kurtosis measure is not robust.

\paragraph{Negentropy} Another measure of the nongaussianity of a random variable is its negentropy:
\begin{align*}
J(y) = H(y_{gauss})-H(y)
\end{align*}
Where $y_{gauss}$ is the Gaussian random variable with the same covariance matrix than $y$. The entropy has the property of being maximal for a Gaussian variable among all variables of equal variance. Thus, the negentropy is zero for a Gaussian variable and nonnegative for other variables. Besides, negentropy is invariant for invertible linear transformations. in this form, the problem consists in maximizing the negentropy.

\paragraph{Preprocessing for ICA}
\begin{itemize}
\item Centering
\item Whitening
\item Dimension reduction (PCA)
\item Band-Pass filtering
\end{itemize}

\section{FastICA}
\paragraph{Idea} Use Gram-Schmidt-like decorrelation to compute independent $w_i$.
\paragraph{Algorithm} Assuming that the data is whitened, initialize $W$ at random and compute for any norm different of the Frobenius norm:
\begin{enumerate}
\item $W=W/\sqrt{||WW^T||}$
\item repeat until convergence: $W=\frac{3}{2}W-\frac{1}{2}WW^TW$
\end{enumerate}

\section{Applications}
\begin{itemize}
\item Separation of Artifacts in MEG Data
\item Finding Hidden Factors in Financial Data (data available on the challenges website)
\item Reducing Noise in Natural Images
\end{itemize}
 
\end{document}

