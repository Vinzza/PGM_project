\documentclass[a4paper]{article}
\usepackage[utf8]{inputenc}
\usepackage[french]{babel}

%% %%%%%%%%%%%%%%%%%%%%%%%%%%%%%%%%%%%%%%%%%%%%%%%%%%%%%%%%%%%%%%%%%%%%%%%%%%%%%
%% %%%%%%% %% %% %  %                                      %  % %% %% %%%%%%%%%%
%%%%% %% %  %                    MES COMMANDES                     %  % %% %%%%%
%%%%%%%%%% %% %% %  %                                      %  % %% %% %%%%%%% %%
%%%%%%%%%%%%%%%%%%%%%%%%%%%%%%%%%%%%%%%%%%%%%%%%%%%%%%%%%%%%%%%%%%%%%%%%%%%%% %%

\newcommand{\zzpackages}[1][french]{
  \usepackage[utf8]{inputenc}
  \usepackage[T1]{fontenc}
  \usepackage[#1]{babel}

  \usepackage{amsthm}
  \usepackage{amsmath}
  \usepackage{amsfonts}
  \usepackage{amssymb}

  \usepackage{xcolor}
  \usepackage{xstring}	%\ifstreqcase
}

%% %%%%%%%%%%%%%%%%%%%%%%%%%%%%%%%%%%%%%%%%%%%%%%%%%%%%%%%%%%%%%%%%%%%%%%%%%%%%%
%                              TESTS / FOR / ...                               %
%%%%%%%%%%%%%%%%%%%%%%%%%%%%%%%%%%%%%%%%%%%%%%%%%%%%%%%%%%%%%%%%%%%%%%%%%%%%% %%

% 
\def\defactive#1#2{
  \catcode`#1=13
  \begingroup
  \lcode`~=`#1
  \lowercase{\endgroup\def~}{#2}
}

\def\zifempty#1#2#3{\def\foo{#1}\ifx\foo\empty\relax#2\else#3\fi}

%% %%%%%%%%%%%%%%%%%%%%%%%%%%%%%%%%%%%%%%%%%%%%%%%%%%%%%%%%%%%%%%%%%%%%%%%%%%%%%
%                           MARGES / HYPERREF / ...                            %
%%%%%%%%%%%%%%%%%%%%%%%%%%%%%%%%%%%%%%%%%%%%%%%%%%%%%%%%%%%%%%%%%%%%%%%%%%%%% %%

\makeatletter
\gdef\@subtitle{}
\def\subtitle#1{\gdef\@subtitle{#1}}

\def\zztitre{
\begingroup\centering
{\bfseries \huge \@title}\par\vspace{.3cm}
\ifx\@subtitle\empty\else{\bfseries \Large \@subtitle}\par\vspace{.5cm}\fi
\Large \@author\par\vspace{.1cm}
\@date\zal\vspace{.3cm}\zal
\zligne\endgroup
}

\makeatother


\newcommand{\zzhyperref}{
\usepackage{hyperref}
\hypersetup{ colorlinks=true, linkcolor=blue!30!black,
citecolor=green!30!black, filecolor=magenta!30!black,
urlcolor=cyan!30!black }
}

\newcommand{\zzmarges}{
  \setlength{\textheight}{620pt}
  \addtolength{\textwidth}{2cm}
  \addtolength{\hoffset}{-1cm}
  \addtolength{\voffset}{-1cm}
  \addtolength{\marginparwidth}{0cm}
  \addtolength{\textheight}{1cm}
} 

\makeatletter
\newcommand{\zzheader}[6]{
\def\@oddhead{\vbox to 0pt{\vss\hspace{0pt} #1\hfill #2\hfill #3\kern4pt\par\kern5pt\hrule height.5pt}}
\def\@oddfoot{\vbox to 0pt{\hrule height.5pt\kern5pt\hbox to \linewidth{\kern4pt {#4}\hss {#5}\hss {#6}\kern4pt}\vss}}}
\makeatother

%%%%%%%%%%%%%%%%%%%%%%%%%%%%%%%%%%%%%%%%%%%%%%%%%%%%%%%%%%%%%%%%%%%%%%%%%%%%%%%%

\newcommand{\zligne}[1][]{
\par\zifempty{#1}%
{\hbox to \linewidth{\leaders\hrule height3pt depth-2.5pt\hfill}}%
{\hbox to \linewidth{\leaders\hrule height3pt depth-2.5pt\hfill\kern.8em #1\kern.8em\leaders\hrule height3pt depth-2.5pt\hfill}}\par
}

%%%%%%%%%%%%%%%%%%%%%%%%%%%%%%%%%%%%%%%%%%%%%%%%%%%%%%%%%%%%%%%%%%%%%%%%%%%%%%%%

\newcommand{\zal}{\par}
\newcommand{\znl}{\zal ~\zal}
\newcommand{\zguill}[2][]{«\,#2\,»}

%% %%%%%%%%%%%%%%%%%%%%%%%%%%%%%%%%%%%%%%%%%%%%%%%%%%%%%%%%%%%%%%%%%%%%%%%%%%%%%
%                             COMMANDES PRATIQUES                              %
%%%%%%%%%%%%%%%%%%%%%%%%%%%%%%%%%%%%%%%%%%%%%%%%%%%%%%%%%%%%%%%%%%%%%%%%%%%%% %%

%% ------- -- -- -  -                                      -  - -- -- --------%%
%---- -- -  -                          ZP                          -  - -- ----%
%%-------- -- -- -  -                                      -  - -- -- ------- %%
\makeatletter
\def\z@first#1#2{#1}
\def\z@second#1#2{#2}
\def\z@zp@selectchar#1#2{
  \IfStrEqCase{#2}{%
    {p}{#1{(}{)}}%
    {c}{#1{[}{]}}%
    {a}{#1{\{}{\}}}%
    {C}{#1{]}{[}}%
    {b}{#1{|}{|}}%
    {n}{#1{\|}{\|}}%
    {i}{#1{[}{]}\!#1{[}{]}}%
    {t}{#1{<}{>}}%
    {v}{#1{.}{.}}%
    {A}{#1{\}}{\{}}%
    {P}{#1{)}{(}}%
    {I}{#1{]}{[}\!#1{]}{[}}%
    {T}{#1{>}{<}}%
  }[#1{(}{)}]%
}

\def\z@zp#1#2\fin#3{
  \z@zp@selectchar{\left\z@first}{#1}#3
  \zifempty{#2}%
        {\z@zp@selectchar{\right\z@second}{#1}}%
        {\z@zp@selectchar{\right\z@second}{#2}}%
}
\newcommand{\zp}[2][]{\zifempty{#1}{\left(#2\right)}{\z@zp#1\fin{#2}}}

\newcommand{\zpbig}[1]{\ifcase#1\relax\vrule width0pt height0pt\or\vrule width0pt height9pt\or\vrule width0pt height10pt\or\vrule width0pt height13pt\else\vrule width0pt height16pt\fi}

%% ------- -- -- -  -                                      -  - -- -- --------%%
%---- -- -  -                   Itemize et autre                   -  - -- ----%
%%-------- -- -- -  -                                      -  - -- -- ------- %%

\newcommand{\zitemize}[1]{
\vspace{-\topsep}\begin{itemize}\setlength\itemsep{0pt plus 1pt}\setlength\parskip{0cm}#1\end{itemize}\vspace{-\topsep}}


%% ------- -- -- -  -                                      -  - -- -- --------%%
%---- -- -  -                        AUTRES                        -  - -- ----%
%%-------- -- -- -  -                                      -  - -- -- ------- %%

\newcommand{\zR}{\mathbb{R}}

\newcommand{\zsum}[2][0pt]{\sum_{\hbox to #1{\hss$\scriptstyle#2$\hss}}}
\newcommand{\zprod}[2][0pt]{\prod_{\hbox to #1{\hss$\scriptstyle#2$\hss}}}

\newcommand{\zseq}[1][=]{\hspace{2pt}\raise .5pt\hbox{\scalebox{.8}{#1}}\hspace{2pt}}

\newcommand{\zop}[2]{\mathrm{#1}\zp{#2}}

\newcommand{\zi}{\mathrm{i}}

\newcommand{\zexp}[1]{\mathrm{e}^{#1}}

\newcommand{\zmatrix}[2]{\left(\begin{array}{#1}#2\end{array}\right)}

\newcommand{\zindic}[1]{%
\hbox to 5.3pt{1\hss l}\hskip -2.5pt\left\{#1\right\}%
}

\newcommand{\zesp}[2][]{%              esperance
\mathbb{E}_{#1}\hskip -3pt\left[\zpbig1\,#2\,\right]%
}

\newcommand{\zprob}[2][]{%             proba
\mathbb{P}_{#1}\hskip -3pt\left(\zpbig1\,#2\,\right)%
}

% Symbole d'indépendance de variable aléatoire
\newcommand{\zindep}{\protect\mathpalette{\protect\z@ind}{\perp}}
\def\z@ind#1#2{\mathrel{\rlap{$#1#2$}\mkern6mu{#1#2}}}


\newcommand{\zdx}[1]{\mathrm{d}#1}

\newcommand{\zderiv}[2]{\frac{\partial #1}{\partial #2}}


\newcommand{\ztr}[2][]{\zifempty{#1}{#2}{\left(#2\right)}^{\hspace{-1pt}\mathsf{T}}\hspace{-1pt}}
\def\zpreind#1#2{ \raise-.35ex\hbox{\scriptsize$#1$}#2}
\def\zpreexp#1#2{ \raise.85ex\hbox{\scriptsize$#1$}#2}

%% %%%%%%%%%%%%%%%%%%%%%%%%%%%%%%%%%%%%%%%%%%%%%%%%%%%%%%%%%%%%%%%%%%%%%%%%%%%%%
%                                 ALGORITHMES                                  %
%%%%%%%%%%%%%%%%%%%%%%%%%%%%%%%%%%%%%%%%%%%%%%%%%%%%%%%%%%%%%%%%%%%%%%%%%%%%% %%

\newcount\z@algo@count
\newdimen\z@algo@indent
\begingroup
  \catcode`\^^M=13             %
  \catcode`\^^I=13             %
  \gdef\z@algo{                %
  \z@algo@count=1
    \begingroup                %
    \catcode`\^^M=13           %
    \def^^M{\leavevmode\par \advance\z@algo@count by 1\z@algo@indent=0pt}%
    \catcode`\^^I=13           %
    \def^^I{\advance\z@algo@indent by 1em}         %
    \everypar{                 %
      \hbox to 0cm{\hss\textcolor{black!30}{\the\z@algo@count~:}}~\kern\z@algo@indent}  %
                               %
    \tt                        %
  }

\endgroup

\newenvironment{zalgo}{\z@algo}{\endgroup}


\makeatother



%% %%%%%%%%%%%%%%%%%%%%%%%%%%%%%%%%%%%%%%%%%%%%%%%%%%%%%%%%%%%%%%%%%%%%%%%%%%%%%
%                                    AUTRE                                     %
%%%%%%%%%%%%%%%%%%%%%%%%%%%%%%%%%%%%%%%%%%%%%%%%%%%%%%%%%%%%%%%%%%%%%%%%%%%%% %%

% epaisseur trait / marge / texte

\def\zfbox#1#2#3{
  \hbox{\vrule width #1
    \vtop{
      \vbox{
        \hrule height #1
        \kern #2
        \hbox{\kern #2 #3\kern #2}
      }%
      \kern #2%
      \hrule height #1
    }%
    \vrule width #1%
  }%
}


\zzpackages
\zzhyperref
\zzmarges

\zzheader{}{}{\today}{De Lara, Tilquin, Vidal}{}{\arabic{page}/\pageref{lastpage}}

\title{Dependence, Correlation and Gaussianity in Independent Component Analysis}
\subtitle{Cardioso -- Résumé}
\author{De Lara, Tilquin, Vidal}
\date{}

%% %%%%%%%%%%%%%%%%%%%%%%%%%%%%%%%%%%%%%%%%%%%%%%%%%%%%%%%%%%%%%%%%%%%%%%%%%%%%%
%                                DEBUT DOCUMENT                                %
%%%%%%%%%%%%%%%%%%%%%%%%%%%%%%%%%%%%%%%%%%%%%%%%%%%%%%%%%%%%%%%%%%%%%%%%%%%%% %%

\newcommand{\Kl}[3][]{\mathrm K_{#1}\!\zp{#2\:\|\:#3}}
\newcommand{\zZ}[2]{\mathrm #1\!\zp{#2}}
\newcommand{\zD}{\mathcal}
\newcommand{\Ng}[2]{\mathcal{N}\zp{#1,\:#2}}

\begin{document} 

\zztitre

\section{Définitions préalables et propriétés}

\subsection{Définitions}
Pour une variable aléatoire $Y\in\zR^n$, on notera $Y_i$ sa $i$-ième composante.
Jusqu'à la fin, pour $X$ une variable aléatoire à valeur dans $\zR^n$, on notera $P\zp{X}$ sa densité de probabilité et $\Sigma_X$ sa matrice de covariance.
\znl

On posera aussi $\zD G$ l'ensemble des distributions gaussiennes, $\zD P$ l'ensemble des distributions \zguill{ produits } (indiquant une indépendance des composantes). Pour une distribution $P$, on définira alors $P^{\zD G}$, $P^{\zD P}$ et $P^{\zD G \wedge \zD P}$ les distributions respectivement gaussiennes, produit et gaussienne produit minimisant la valeur de leur divergence par rapport à $P$. On verra ces distributions comme des projections sur ces différents espaces.\znl

On définit alors les grandeurs suivantes.\znl


La \textbf{divergence de Kullback--Leibler} de la distribution $Q$ par rapport à $P$ :
\begin{equation}
  \Kl PQ = \int_{\zR^n} P(x) \log\frac{P(x)}{Q(x)} \zdx x.
\end{equation}

L'\textbf{entropie} de $Y$ :\begin{equation}
  \zZ HP = - \int_{\zR^n} P(x)\log P(x) \zdx x.
\end{equation}

L'\textbf{information mutuelle}, que l'on prendra pour mesure d'indépendance :
\begin{equation}
  \zZ IY = \Kl {\zpbig2 P(Y)}{\Pi_i P_i(Y_i)} = \Kl {\zpbig2 P(Y)}{P(Y)^{\zD P}}.
\end{equation}

La \textbf{Non-Gaussianité} de $Y$ :
\begin{equation}
  \zZ GY = \Kl{Y}{\Ng{\zesp Y}{\Sigma_Y}\zpbig2} = \Kl {\zpbig2 P(Y)}{P(Y)^{\zD G}}.
\end{equation}

La \textbf{Corrélation} de $Y$ :
\begin{equation}\begin{array}{rcl}
  \zZ CY &=& \displaystyle\Kl{\Ng{\zesp Y}{\Sigma_Y}\zpbig2}{\Ng{\zesp Y}{\mathrm{Diag}\:\Sigma_Y}}\\
  &=&\Kl{\zpbig2 P(Y)^{\zD G}}{P(Y)^{\zD P\wedge\zD G}}\\
  &=& \displaystyle\frac 12\log\frac{\det\zp{\mathrm{Diag}(\Sigma_Y)}}{\det\zp{\Sigma_Y}}.
\end{array}\end{equation}

Dans la suite, on se permettra, pour éviter des notation trop lourde, d'écrire l'information mutuelle, la non-gaussianité et la corrélation d'une distribution.

%% --------- ---- -  -             PROPRIETES             -  - ---- --------- %%

\subsection{Propriétés}

Par propriété, si $\Kl PQ=0$, les distributions $P$ et $Q$  sont égales sur les espaces de mesures non nulles.
Remarquons la propriété suivante, si $T$ est une matrice inversible et $\mu$ est un vecteur quelconque, on a :
\begin{equation}
  \Kl{\zpbig2P(Y)}{P(Z)} = \Kl{\zpbig2P(\mu+TY)}{P(\mu+TZ)}.
\end{equation}\znl

Par définition, $\zZ IY = 0$, on aura $P(Y) = \prod_i P_i(Y_i)$ et les composantes de $Y$ seront indépendantes.
On a la relation, d'où découle l'égalité de la définition de l'indépendance mutuelle \[
\Kl{P(Y)}{\zpbig2\prod_iQ_i} = \zZ IY + \sum_i \Kl{\zpbig2P(Y_i)}{Q_i}.
\]

Si $T$ est une matrice inversible et $\mu$ est un vecteur quelconque, on a l'égalité suivante :
\begin{equation}
  \zZ G{\zpbig2P(Y)} = \zZ G{\zpbig2P(\mu+TY)}.
\end{equation}

%% --------- ---- -  -             RELATIONS              -  - ---- --------- %%

\subsection{Relations}

On a les relations suivantes, découlant de la définition de la divergence de Kullback--Leibler ou du \zguill{théorème de Pythagore} :
\begin{equation}
  \zZ IY = \sum_i \zZ H{P_i} - \zZ H{P},
\end{equation}
\[
\Kl{\zpbig2 P(Y)}{P(Y)^{\zD P\wedge\zD G}} = \zZ IY + \sum_i \zZ G{Y_i},
\]
\[
\Kl{\zpbig2 P(Y)}{P(Y)^{\zD P\wedge\zD G}} = \zZ GY + \zZ CY,
\]

\begin{equation}
        \zfbox{1pt}{5pt}{$\displaystyle\kern1cm\zZ IY + \sum_i\zZ G{Y_i} = \zZ GY + \zZ CY.\kern1cm$}\label{eqn:1}
\end{equation}

Cette dernière relation nous donne que minimiser la grandeur $I\zp{Y}$ revient à minimiser $\zZ CY - \sum_i\zZ G{Y_i}$, la Non-gaussianité de $Y$ étant indépendante ici du changement de référentiel que l'on recherche.



%% ------- -- -- -  -                                      -  - -- -- --------%%
%---- -- -  -                      Géométrie                       -  - -- ----%
%%-------- -- -- -  -                                      -  - -- -- ------- %%

\section{Géométrie}
\subsection{Théorème de Pythagore}
On considère ici les deux variétés $\zD P$ et $\zD G$, comme définies précédemment. Remarquons que ces deux variétés sont toutes les deux des familles exponentielles et ainsi, elles vérifient toutes les deux la propriétés suivantes. Si on y prend deux distributions $p$ et $q$, alors toutes les distributions du segment exponentiel qu'elles définissent y appartiennent aussi, le segment exponentiel étant défini par \[
w_{\alpha}(x) = p(x)^{1-\alpha}q(x)^\alpha\zexp{-\psi(\alpha)},
\]
avec $\psi(\alpha)$ un coefficient de normalisation.

On peut montrer, par ailleurs, que la famille exponentielle $\zD G$ est de dimension\linebreak$L_{\zD G}=n + \frac 12 n(n+1)$ dans le sens où on peut trouver une mesure de référence $g(x)$ (non nécessairement une distribution) et une \zguill{base} de $L$ fonctions scalaire $S_l(x)$ telle que toute distribution s'écrive sous la forme :\[
p_\alpha(x) = g(x)\exp\!\zp{\sum_l \alpha_lS_l(x) - \psi(\alpha)},\]
pour $\psi(\alpha)$ une fonction de normalisation et $\alpha\in\zR^L$.
\znl

Remarquons que dans le cas de $\zD G$, on peut prendre comme \zguill{base} les fonctions $y\mapsto y_i$ et $y\mapsto y_iy_j$ pour $i$ et $j$ dans $\zp[i]{1,n}$.
\znl

Le Théorème de Pythagore s'énonce de la manière suivante :\\
Si $\zD E$ est une famille exponentielle et $P$ est une distribution quelconque, il existe alors une unique distribution $P^{\zD E}$ de $\zD E$ vérifiant :\[
\forall Q\in\zD E,\qquad\Kl PQ = \Kl P{P^{\zD E}} + \Kl {P^{\zD E}}Q.
\]
On peut voir $P^{\zD E}$ comme la \zguill{projection orthogonale} de $P$ sur la famille $\zD E$. Par positivité de la divergence de Kullback-Leibler, on peut voir $P^{\zD E}$ comme la distribution de $\zD E$ minimisant sa divergence par rapport à $P$.\znl

On peut alors remarquer que l'équation \ref{eqn:1} peut être interprété comme suit : on obtient le même résultat en projetant $P$ d'abord sur $\zD G$ puis sur $\zD P\wedge\zD G$ ou d'abord sur $\zD P$ puis sur $\zD P\wedge\zD G$.
Remarquons que tous les termes de cette équations sont invariants par translation et changement d'échelle indépendamment sur les différentes coordonnées. L'espace $\zD G\wedge\zD P$ étant de dimension $2n$ (chaque coordonnée correspond à une gaussienne avec $2$ paramètres), qui correspond exactement à la dimension de l'ensemble des transformations précédentes, tout se ramène exactement au cas d'une seule distribution dans $\zD G\wedge\zD P$.

\subsection{Structures marginales}

On va ici s'intéresser à l'espace \zguill{union} des deux variétés $\zD G$ et $\zD P$ que l'on notera $\zD G\vee\zD P$ qui correspondra à la plus petite famille exponentielle qui contient $\zD G$ et $\zD P$, ce qui revient à considérer exactement toutes les distributions de la forme \begin{equation}\label{eqn:3}
p(y) = \phi(y)\exp\zp{\sum_{l=1}^{L_{\zD G}}\alpha_l S_l(y) + \sum_{i=1}^{n}r_i(y_i) - \psi},
\end{equation}
avec $S_l$ une \zguill{base} de $\zD G$, $r_i$ des fonctions réelles (puisque l'on a pas de \zguill{base} finie pour $\zD P$), $\psi$ un coefficient de normalisation et $\phi(y)$ une distribution normale homogène ($\Ng{0}{I_n}$).

Par le théorème de Pythagore, on trouve que pour toute distribution $Q$ de $\zD G$ ou de $\zD P$, on a :\begin{equation}\label{eqn:2}
\Kl PQ = \Kl P{P^{\zD G\vee\zD P}} + \Kl {P^{\zD G\vee\zD P}}Q.
\end{equation}
Ainsi, la divergence minimum sur $\zD G$ par rapport à $P$ et par rapport à $P^{\zD G\vee\zDP P}$ est atteinte au même point $P^{\zD G}$, ce qui revient à dire que :\[
\zp{P^{\zD G\vee \zD P}}^{\zD G} = P^{\zD G},
\]
de même pour $P^{\zD P}$.

Ainsi, en reprenant l'équation \ref{eqn:2} et en l'utilisant avec $Q=P^{\zD P}$ et avec $Q=P^{\zD G}$, on trouve les relations suivantes : \[\begin{array}{rcl}
\zZ IP &=& \Kl P{P^{\zD G\vee\zD P}} + \zZ I{P^{\zD G\vee\zD P}}\\
\zZ GP &=& \Kl P{P^{\zD G\vee\zD P}} + \zZ G{P^{\zD G\vee\zD P}}\\
\end{array}\]

En supposant que les valeurs des divergences sont suffisamment petites, on peut supposer que la figure formé par $P^{\zD G\vee\zD P}$, $P^{\zD G}$, $P^{\zD P}$ et $P^{\zD G\wedge\zD P}$ et un rectangle. L'égalité des longueurs nous donne alors :
\[
\zZ I{P^{\zD P\vee\zD G}} \simeq \zZ CP\qquad\mbox{et}\qquad
\zZ G{P^{\zD P\vee\zD G}} \simeq \sum_i\zZ G{P_i}.\]
Cela donne, avec les équations précédentes :\[\begin{array}{rcl}
\zZ IP &\simeq& \Kl P{P^{\zD G\vee\zD P}} + \zZ CP\\
\zZ GP &\simeq& \Kl P{P^{\zD G\vee\zD P}} + \sum_i\zZ G{P_i}\\
\end{array}\]
La distribution $P^{\zD G\vee\zD P}$ peut être interprété comme la distribution la plus simple approchant la distribution $P$, dans le sens où elle capture la structure marginale de $P$ et sa structure de premier et second ordre (voir équation \ref{eqn:3}).

%% ------- -- -- -  -                                      -  - -- -- --------%%
%---- -- -  -                       Cumulant                       -  - -- ----%
%%-------- -- -- -  -                                      -  - -- -- ------- %%

\section{Cumulant et géométrie locale}

Pour rappel, les cumulants $\kappa_n$ d'une variable aléatoire $X$ sont définis avec la fonction génératrice des cumulants :\begin{equation}
  g(t) = \log\zesp{\zexp{tX}} = \sum_{n=1}^{\infty} \frac{\kappa_n}{n!}t^n
\end{equation}

On s'intéresse ici aux distributions au voisinage de $P^{\zD G\wedge\zD P}$, ce qui correspond aux distributions faiblement corrélées et faiblement non-gaussiennes. On assimilera les variétés $\zD P$ et $\zD G$ à leur plan tangent et la divergence de Kullback-Leibler à une mesure quadratique.

\subsection{Construction des plans tangents}

Pour deux distributions $p(x)$ et $n(x)$, on définit la fonction \[
e_p(x) = \frac{p(x)}{n(x)}-1,
\]
qui sera alors d'espérance nulle selon $n(x)$ : $\zesp[X\sim n]{e_p(X)}=0$.

On cherche alors à identifier les distributions $p$ proches de $n$ aux \zguill{petites} fonctions d'espérance nulle selon $n$.
Si on considère une distribution proche de $n$, la fonction $e_p$ sera $petite$ et d'espérance nulle. Réciproquement, pour une fonction $e_p$ données, on considérera alors $p(x)=n(x)\zp{e_p(x)+1}$ qui sera bien une distribution proche de $n(x)$.

Ainsi, on peut identifier l'espace vectoriel des variables aléatoires d'espérance nulle et de variance finie au plan tangent à la variété des distributions au point $n$.
\znl

Si on considère deux distributions $p$ et $q$ proches de $n$, on peut exprimer $\mathrm{K}_n$ l'expansion au second ordre suivant $e_p$ et $e_q$ de $\Kl pq$ de la manière suivante :\begin{equation}
  \Kl[n]pq = \frac 12 \zesp[X\sim n]{\zp{e_p(X)-e_q(X)}^2}
\end{equation}

\subsection{Expansion de Gram-Charlier}


\subsection{Base d'Hermite}
\subsection{Approximation de la divergence par des petits cumulants}
\subsection{Décomposition en quatres parties}
\subsection{Divergences et objectifs}
%% %%%%%%%%%%%%%%%%%%%%%%%%%%%%%%%%%%%%%%%%%%%%%%%%%%%%%%%%%%%%%%%%%%%%%%%%%%%%%
%                               FIN DU DOCUMENT                                %
%%%%%%%%%%%%%%%%%%%%%%%%%%%%%%%%%%%%%%%%%%%%%%%%%%%%%%%%%%%%%%%%%%%%%%%%%%%%% %%


\bibliographystyle{plain}
\bibliography{Biblio}{}
\nocite{*}

\label{lastpage}

\end{document}
